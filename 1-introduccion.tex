\chapter{Introducción}

El proyecto que explica esta memoria se encuadra dentro del manejo, control, recogida y procesado de datos de sensores y actuadores de un drone a distancia. El drone es un vehículo aéreo no tripulado al que podemos definir dentro de la robótica aérea. En las siguientes páginas se dará unas pinceladas sobre la robótica, su historia y uso actual. También hablaremos sobre los sistemas actuales de control y manejo del drone, y para finalizar daremos una visión global sobre las tecnologías existentes dentro de las Comunicaciones en Tiempo Real (RTC, Real Time Communications).\\

\section{Robótica}

\section{Sistemas de control de drones}

Los drones pertenecen a la rama de robótica aérea, y a su vez dentro de los vehículos aéreos es un vehículo aéreo no trupilado (UAV, Unmanned  Aerial Vehicle).Es un vehículo no tripulado, pero no autónomo, por lo que necesitan ser teleoperados desde tierra. Los sistemas actuales para ello se pueden dividir en dos grupos, los controlados mediante radiofrecuencia y los que usan sistemas propietarios.\\

\subsection{Radiocontrol}

Es la técnica que permite el gobierno de un objeto a distancia de manera inalámbrica mediante una emisora de control remoto. Por otra parte, a bordo del vehículo, en nuestro caso un drone, debe ir una receptora de radio control. \\
***IMAGEN***
La comunicación entre receptor y transmisor se efectúa mediante radiofrecuencia, existiendo diferentes sistemas de emisión, como AM, FM o 2.4Ghz con diferentes tipo de codificación, PCM, PPM…\\

Estos sistemas tienen varias limitaciones. Una es el número de canales máximo del sistema, ya que se usa un canal para cada elemento de control disponible: elevación, giro, rotación… La segunda y posiblemente más crítica son las interferencias. Si se producen interferencias ya sea por ruido o por varios emisores trabajando en las cercanías se puede perder el control de la aeronave y destruir la misma o incluso dañar a personas.\\
 
\subsection{Sistemas alternativos}

En la actualidad a parte del radiocontrol tenemos el control a través de WiFi. Este sistema consiste en la creación de una red WiFi por parte del drone a la cual se conecta el dispositivo con el que se maneja. Este dispositivo puede ser un mando diseñado y comercializado por la propia marca o un dispositivo móvil, el cuál usa una aplicación que es la que gestiona la conexión y transferencia de datos.\\

Empresas como DJI usa sistemas mixtos que consisten en teleoperar el drone vía radiocontrol, pero la gestión de la cámara, con visualización incluida.\\
 
La empresa francesa Parrot, la cual tiene una flota de diversos modelos de drones, utiliza un sistema de red WiFi, a la cuál conectas un dispositivo móvil ya sea Android o iOS, y mediante una aplicación desarrollada por ellos se puede controlar el drone, así como tener otras funcionalidades como la grabación de video, captura de imágenes y gestión de parámetros de vuelo como altura o velocidad máxima. En capítulos posteriores profundizaremos en este sistema implementado, ya que es el que usaremos para desarrollar el nuestro.\\
 
\section{Tecnologías de comunicación a distancia en tiempo real}

\section{Motivación}
