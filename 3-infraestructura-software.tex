\chapter{Infraestructura Software} 

Una vez presentados los objetivos que tenemos marcados hay que echar una mirada a las tecnologías software que tenemos disponibles para utilizarlas como base y pilares del proyecto. La más importante y sobre la que está centrado el proyecto es WebRTC. Luego utilizaremos, como apoyo para cubrir las partes que WebRTC no llega, ICEJS junto con la infraestructura que tiene ya desarrollada JdeRobot.


\section{WebRTC}

WebRTC es un proyecto opensource y gratuito que nos permite tener en el navegador tecnología en tiempo real ('Real-Time Communication' ó RTC), sin plugins, a través de una simple API de JavaScript. Facilita las llamadas de voz, videollamadas, chat y compartimiento de archivos y datos. WebRTC es una tecnología Peer-to-Peer, por lo que nos permite desarrollar estas aplicaciones para que funcionen directamente desde un navegador a otro sin pasar por servidor intermedio. \\

En todo el documento nos referimos a una llamada WebRTC entre dos navegadores, pero hay que tener en cuenta que en muchas ocasiones la conexión puede ser entre un navegador y algún tipo de servidor ó MCU configurado y trabajando con la API de WebRTC para ofrecernos cualquier servicio en tiempo real.\\

La API de WebRTC nos proporciona todo el set completo de funciones para manejar y crear nuestras aplicaciones, como el control y administración de la conexión, codificación/decodificación del audio/vídeo, negociación entre navegadores, control de la conexión, firewall y NAT traversal.\\

Que WebRTC no necesita un servidor no es del todo cierto, ya que sí que necesita de lo que llamamos Servidor de Señalización. Este es el encargado de establecer el primer contacto entre ellos, facilitando el intercambio de paquetes de la negociación WebRTC.\\

WebRTC está compuesto de 3 API's:

\begin{itemize}
\item getUserMedia
\item RTCPeerConnection
\item RTCDataChannel
\end{itemize}

A continuación vamos a desgranar y explicar como funciona el intercambio de paquetes de señalización así como cada una de las API's que componen WebRTC.

\subsection{Señalización} 

Señalización es el proceso de intercambio de datos y metadatos necesarios para coordinar una llamada entre navegadores con WebRTC. Para realizar esta labor WebRTC necesita de la ayuda de un servidor externo ya que la norma deja el campo de la señalización a la capa de la aplicación.\\

Entre las labores de la señalización se encuentran la detección de los peers, el intercambio de paquetes de control de la sesión como los \textit{ICE candidates} y los \textit{SDP (Session Description Protocol)}, las prestaciones que puede darnos cada peer así como cualquier otro dato o paquete necesario para realizar este 'apretón de manos' inicial.\\

WebRTC no especifica que tipo de servidor hemos de usar para estas funciones. Esto es debido a que diferentes aplicaciones pueden preferir diferentes servidores básicos o personalizados según sus necesidades. La única restricción es el uso de la arquitectura JSEP, la cuál especifica cómo debe ser la secuencia de señalización para tener una llamada exitosa.

\subsubsection{Arquitectura \textit{JSEP (JavaScript Session Establishment Protocol)}}

El servidor debe usar la arquitectura JSEP. Esta arquitectura elimina al navegador de casi todo el flujo de señalización, el cual se maneja desde JavaScript haciendo uso de dos interfaces: transfiriendo los SDP local/remoto e interactuando con la maquina de estados ICE. Esta arquitectura nos evita, entre otras cosas, que el navegador tenga que guardar estados de sesión, de tal manera que se pueden guardar en el servidor y evitar problemas si la página se recarga, por ejemplo. \\

Como ya hemos comentado, JSEP no establece un modelo particular de señalización más allá de usar uno capaz de realizar el intercambio de los SDP y ICE según la norma RFC3264 de oferta/respuesta, de tal manera que ambas partes de la llamada sepan como actuar en cada momento. JSEP nos da los mecanismos necesarios para crear estas ofertas, así como aplicarlas a las sesión.\\

El orden en que se llaman a estas mecanismos o funciones de la API es importante, por lo que la aplicación deberá saber el orden en el que tiene que llamar a cada una, convertir las ofertas en mensajes que entienda el protocolo de señalización elegido y hacer la conversión inversa con los mensajes que se reciben para obtener ofertas que entiendan as API's.\\

El manejo de las \textit{session descriptions} es simple y sencillo. Siempre que el intercambio de una oferta/respuesta es necesario, el peer que establece la llamada ó \textit{caller} crea la oferta llamando a la función \emph{createOffer()} de la API. Esta oferta puede ser modificada por la aplicación si así fuese necesario y se establece como configuración local en ese peer con \emph{setLocalDescription()} y se envía al peer remoto a través del servidor de señalización utilizado. Al recibir esta oferta el peer \textit{called} lo utiliza como configuración del otro peer con \emph{setRemoteDescription()} y utiliza \emph{createAnswer()} para crear una respuesta apropiada, la cual establece como configuración local (\emph{setLocalDescription()}) y envía la respuesta de vuelta a través del servidor de señalización. El caller al recibir la respuesta llama también a \emph{setRemoteDescription()}, y de esta manera ambos lados tienen la información del media propia y la del peer remoto.\\


\subsubsection{Descriptores de Sesión y Máquina de Estados}

Para establecer un intercambio de media, el \textit{user agent} del navegador necesita parámetros específicos para indicar al peer remoto qué es lo que va a transmitir, de la misma manera que necesita saber el media que va a recibir para saber como decodificarlo y manejarlo. Estos datos se determinan en la descripción de sesión (SDP), los cuales se intercambian en ofertas/respuestas usando las API's JSEP como ya hemos visto anteriormente.\\

Si el SDP pertenece a la parte local o remota tiene su importancia. Una vez realizado el intercambio, cada parte mirará la lista de codecs soportados por él mismo y por la otra parte, y el cruce de los resultados determinará que codecs debe enviar y cuál espera recibir. Como podemos intuir, los parámetros exactos de la transmisión solo se pueden saber una vez la oferta y la respuesta han sido intercambiados. Sin embargo, hay ocasiones en las que el caller o el que hace la oferta puede recibir media después de enviar esta pero antes de recibir la respuesta de la otra parte. Para procesar este media de manera adecuada, el manejador de caller debe conoce los detalles de la oferta antes de que la respuesta llegue.\\

Por lo tanto, para manejar los session description de manera correcta, los user agent necesitan:

\begin{itemize}
\item Conocer si el session description pertenece a la parte local o remota.
\item Conocer si el session description es una oferta o una respuesta.
\item Permitir a la oferta ser especificada independientemente de la respuesta.
\end{itemize}

Para satisfacer estas premisas JSEP aborda esto añadiendo los métodos setLocalDescription() y setRemoteDescription() y teniendo un campo en los session description indicando el tipo de sesión que se suministra.\\

JSEP también permite el uso de respuestas provisionales. Estas respuestas permiten al peer remoto o called comunicar e informar de los parámetros iniciales de la sesión al caller, de tal manera que la sesión puede comenzar mientras se espera una respuesta final posteriormente. Este concepto es importante en el modelo oferta/respuesta, ya que al recibir una de estas respuestas el caller puede liberar y usar más recursos como extra \textit{ICE candidates, TURN candidates} o vídeo decodecs. Estas respuestas provisionales no provocan ningún tipo de des-asignación o problema, por lo que pueden ser recibidas a lo largo de la llamada para estabilizar o mejorar la misma según varíen las condiciones del ancho de banda de uno de los peers, por ejemplo.\\


DIAGRAMAS CON LA MAQUINA DE ESTADOS (pagina rtcweb-wg.github.io)
LISTA DE CODECS PERMITIDOS



http://www.html5rocks.com/en/tutorials/webrtc/infrastructure/
https://www.webrtc-experiment.com/docs/WebRTC-Signaling-Concepts.html

\subsubsection{Formato de los Descriptores de Sesión}

En la especificación WebRTC, los descriptores de sesión o session descriptions están formados por mensajes \textit{SDP (Session Description Protocol)}. Este formato no es el más óptimo para manipular con JavaScript, pero es el más popular y aceptado en el campo de las comunicaciones audiovisuales en tiempo real. Este formato es el que usa JSEP para formar e intercambiar los descriptores de sesión.\\

Para facilitar el procesado en JavaScript y una futura flexibilidad, los SDP nos los genera la API como un objeto o \textit{blob}. Si en un futuro WebRTC soporta algún formato nuevo para los descriptores de sesión, estos serán fácilmente añadidos y habilitados para poder usarlos en nuestras aplicación en vez de SDP.\\

La forma que tiene un paquete SDP es la siguiente:

\begin{lstlisting}[caption=Ejemplo paquete SDP]
v=0
o=- 7729291447651054566 1 IN IP4 0.0.0.0
s=-
t=0 0
a=group:BUNDLE a1 d1
a=ice-options:trickle
m=audio 9 UDP/TLS/RTP/SAVPF 96 0 8 97 98
c=IN IP4 0.0.0.0
a=rtcp:9 IN IP4 0.0.0.0
a=mid:a1
a=msid:QI39StLS8W7ZbQl1sJsWUXkr3Zf12fJUvzQ1
       QI39StLS8W7ZbQl1sJsWUXkr3Zf12fJUvzQ1a0
a=sendrecv
a=rtpmap:96 opus/48000/2
a=rtpmap:0 PCMU/8000
a=rtpmap:8 PCMA/8000
a=rtpmap:97 telephone-event/8000
a=rtpmap:98 telephone-event/48000
a=maxptime:120
a=ice-ufrag:7sFvz2gdLkEwjZEr
a=ice-pwd:dOTZKZNVlO9RSGsEGM63JXT2
a=fingerprint:sha-256 6B:8B:F0:65:5F:78:E2:51:3B:AC:6F:F3:3F:46:1B:35
                     :DC:B8:5F:64:1A:24:C2:43:F0:A1:58:D0:A1:2C:19:08
a=setup:active
a=rtcp-mux
a=rtcp-rsize
a=extmap:1 urn:ietf:params:rtp-hdrext:ssrc-audio-level
a=extmap:2 urn:ietf:params:rtp-hdrext:sdes:mid
a=ssrc:4429951804 cname:Q/NWs1ao1HmN4Xa5

m=application 9 UDP/DTLS/SCTP webrtc-datachannel
c=IN IP4 0.0.0.0
a=mid:d1
a=fmtp:webrtc-datachannel max-message-size=65536
a=sctp-port 5000
a=fingerprint:sha-256 6B:8B:F0:65:5F:78:E2:51:3B:AC:6F:F3:3F:46:1B:35
                     :DC:B8:5F:64:1A:24:C2:43:F0:A1:58:D0:A1:2C:19:08
a=setup:active
\end{lstlisting}


\subsubsection{Interactive Connectivity Establishment (ICE)}

Al igual que los peer tienen que intercambiar información sobre el media, también necesitan hacerlo sobre la información de \textit{network} para que los peer sean visibles entre ellos y puedan alcanzarse. ICE es una técnica usada en aplicaciones de voz, vídeo, peer-2-peer, entre otros que nos permite solucionar problemas de alcance de red entre dos ordenadores. Estos problemas son debidos a que los ordenadores suelen estar dentro de una red privada y/o firewall. Esta técnica nos permite descubrir suficiente información sobre la topología de los otros peer para encontrar una o varias rutas potenciales entre ellos.\\

Esta información ha de obtenerse de manera local en cada peer con el \textit{ICE Agent} asociado a cada objeto RTCPeerConnection. El ICE Agent es responsable de: 

\begin{itemize}
\item Reunir tuplas candidatas de IP + Puerto.
\item Realizar pruebas de conectividad entre los peers.
\item Enviar \textit{keepalives}.
\end{itemize}

Una vez se ha finalizado y configurado el proceso de session description, el ICE Agent local comienza automáticamente el proceso de descubrir todos los posibles candidatos en el peer local. Cada candidato posible se le llama \textit{ICE Candidate}:

\begin{enumerate}
\item El ICE Agent pide al sistema operativo las direcciones IP locales.
\item Consulta a un servidor \emph{STUN (Session Traversal Utilities for NAT}) externo la tupla de dirección IP pública y puerto del peer.
\item Consulta a un servidor \emph{TURN (Traversal Using Relays around NAT)} como último recurso. 
\end{enumerate}

Como podemos ver, ICE necesita de servidores externos para obtener la tupla de dirección IP y puerto públicos necesarios para el otro peer si esta fuera de la misma red local. STUN  es un protocolo estandarizado para descubrir direcciones IP publicas de equipos que están detrás de un NAT. TURN es un servidor para transmitir mensajes entre dos clientes. Este servidor solo se usará si falla la conexión Peer-2-Peer después de probar con las direcciones IP locales y las públicas obtenidas en el servidor STUN. No es obligatorio configurar estos servidores. Si la conexión entre los peer es en la misma red no necesitamos configurar servidores STUN/TURN ya que con las direcciones locales es suficiente.\\

DIAGRAMA DE ESTADOS DE ICE (PAGINA 132 (118) DEL LIBRO DE WEBRTC)

Como podemos ver en la figura, cada vez que el navegador recolecta un nuevo ICE Candidate, la función \textit{handleIceCandidate(}) se activa y es la encargada de enviar el candidato al peer remoto a través del servidor.\\
 
Cuando un ICE Candidate llega al peer remoto, se añade en RTCPeerConnection la información de sesión que ontiene ese paquete con \textit{setRemoteDescription()}, de tal manera que el ICE Agent puede empezar a hacer pruebas de conectividad para ver si puede alcanzar al otro peer.\\

Una vez los dos ICE Agent tienen una lista completa de los ICE Candidates de ambos peers, cada agente comprueba pareando ambas lista cuales funcionan. Para ello tienen una planificación de prioridades: primero direcciones IP locales, luego IP públicas y finalmente si ambas fallan servidor TURN. Cada comprobación es una peticion/respuesta STUN que el cliente realiza con un particular candidato enviando una petición STUN desde el candidato local al candidato remoto.\\

Si uno de los pares de candidatos funciona, entonces tenemos una ruta de conexión entre ambos peer. Si todos los candidatos fallan ambas conexiones RTCPeerConnection se marca como fallida o la conexión se hace a través de un servidor TURN.\\

Cuando una conexión se ha establecido correctamente cada ICE Agent continua haciendo peticiones STUN periódicas al otro peer, lo cual sirve también como \textit{keepalives}.\\

\begin{normalsize}
\noindent \textbf{ICE Candidate Trickling}\\
\end{normalsize}

ICE Candidate Tricking es una técnica por la cuál el caller puede incrementar el número de candidatos para el called después de la primera oferta. Este proceso permite al called comenzar a establecer las conexiones ICE por encima de la llamada inmediatamente, sin tener que esperar a que el calle recopile todos los posibles candidatos. Con esta técnica conseguimos un establecimiento del media más rápido.\\

Esta técnica es opcional. Las aplicaciones que lo soportan pueden enviar directamente la oferta inicial inmediatamente y enviar candidatos individuales cuando los vayan teniendo; las aplicaciones que no lo soportan simplemente esperan la indicación de que el \textit{gathering} está completo, crear la oferta con todos los candidatos y enviarla.\\


\begin{normalsize}
\noindent \textbf{Formato de un ICE Candidate}\\
\end{normalsize}

\begin{lstlisting}[caption=Ejemplo paquete SDP]
candidate:1 1 UDP 1694498815 192.0.2.33 10000 typ host
\end{lstlisting}

\noindent \textbf{Fundación (1):} Identificador para cada candidato del mismo tipo, misma interfaz y servidor STUN.\\
\textbf{ID (1):} Identificador. 1 para RTP, 2 para RTCP.\\
\textbf{Protocolo (UDP):} Protocolo de transporte del candidato.\\
\textbf{Prioridad (1694498815): }Prioridad del componente dado.\\
\textbf{Dirección IP y puerto (192.0.2.33 10000): }Direccion IP y puerto del candidato.\\
\textbf{Tipo (typ):} Tipo del componente.\\
\textbf{Dirección relacionada (host):} Información opcional que contiene dirección IP y puerto privado.\\

\subsubsection{Negociación de Vídeo}

La negociación de vídeo es un proceso a través del que el cada peer puede indicar al otro que resoluciones y \textit{frames rate} de vídeo es capaz de recibir. Esto lo hace a través de un atributo \textit{"a=imageattr"} en el SDP. Cada peer puede tener limites como la capacidad de proceso que el decoder tiene, o simplemente restricciones de la aplicación.\\


\subsection{WebRTC API's}

\subsubsection{getUserMedia} 

getUserMedia es la API encargada de suministrarnos el streaming de audio y/o vídeo. Pide permiso al usuario para acceder y utilizar los dispositivos hardware como la cámara y el micrófono. Por el momento solo esta disponible para captar el hardware de audio y vídeo anteriormente mencionado, pero se pretende mejorar y ampliar la API para que en un futuro se pueda hacer streaming de casi cualquier fuente de datos, como un disco o sensores conectados al ordenador.\\

Cuándo hacemos una llamada a la función \textit{getUserMedia()} nos devuelve un objeto que contiene las pistas de audio y/o vídeo. Para llamar a esta función necesitamos suministrarle las \textit{constraints} o restricciones. En esta variable le indicamos si queremos audio, vídeo, o ambas, así como las propiedades que tienen que tener estos: resolución mínima, máxima o ideal del vídeo, framerate o cámara a usar en un dispositivo móvil (frontal/trasera) son algunas de las restricciones que podemos configurar según nuestras necesidades.\\

La llamada a la función nos devuelve un \textit{callback}. Si hay éxito al llamar la función nos devuelve el objeto anteriormente mencionado y pasa a ejecutar la función callback de éxito. En caso contrario nos devuelve el código del error y llama a la función callback de fallo. Esta forma de llamar a la función desaparecerá en próximas revisiones ya que se está implementando una llamada por \textit{promises}.\\


\subsubsection{RTCPeerConnection}

Esta API es la encargada de crear la conexión Peer-2-Peer entre el navegador local y el remoto. Trabaja de manera diferente si es el que hace la llamada (\textit{caller}) y el que la recibe (\textit{called}) debido a la señalización oferta/respuesta que ya hemos visto. Para ello hace uso de sus funciones para completar el proceso de señalización descrito anteriormente.\\

\noindent Es también la encargada de manejar la conexión una vez establecida. Entre las funciones automáticas que realiza se encuentran:\\

\begin{enumerate}
\item Ocultamiento de paquetes perdidos.
\item Cancelación de eco.
\item Adaptación del ancho de banda.
\item Buffer dinámico en función del jitter o delay.
\item Control automático de ganancia.
\item Reducción o eliminación de ruido.
\end{enumerate}

Es el desarrollador de la aplicación el encargado de llamar a las funciones que componen esta API en el orden, tiempo y forma correcta para cumplir con la arquitectura JSEP y conseguir un intercambio de descriptores de sesión y de ICE Candidates exitoso.\\

\noindent La función admite dos argumentos, los cuales son opcionales: 

\begin{lstlisting}[caption=Llamada a función RTCPeerConnection]
var PC = new RTCPeerConnection(ICEconfig, pcConstraints);
\end{lstlisting}

\textit{ICEConfig} es una variable la cuál contiene los datos necesarios para conectarse con el servidor STUN y TURN y poder hacer NAT Trasversal. \textit{pcConstraints} también es una variable a la que se le pueden añadir una serie de restricciones como \textit{RtpDataChannels}, la cuál estará obsoleta en futuras versiones y se dejará de usar. Esta variable está para futuras mejoras.\\

\subsubsection{RTCDataChannel}


\section{ICE y ICEJS}

df

\section{JdeRobot}

dfdfdf

\section{ArDrone de Parrot}

