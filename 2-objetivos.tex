\chapter{Objetivos}

Una vez situado el Trabajo Fin de Grado en su contexto vamos a presentar los objetivos que nos marcamos para la realización del mismo. Los objetivos a abordar son tres, los cuáles expongo a continuación.

\section{Objetivos}
\subsection{Conexión local}

Como primer objetivo tenemos que desarrollar una conexión directa con el drone. Esta conexión tiene que ser bidireccional, ya que tenemos que acceder a los sensores y actuadores del drone, así como a su cámara, pero también es necesario mandarle ordenes de movimiento.\\

Como requisito fundamental para esta conexión es que tiene que ser en tiempo real y que sea lo suficientemente fluida y ligera como para que no introduzca ningún tipo de retardo.\\


\subsection{Conexión remota}

El segundo punto con el que nos enfrentamos es utilizar una tecnología web moderna y actual con la que poder teleoperar el drone. Esta tecnología tiene que ser lo suficientemente versátil como para poder implementarla en nuestro proyecto. Por otro lado, al igual que la conexión local, tiene que ser bidireccional, poder transportar audio y vídeo e introducir el mínimo retraso en la comunicación posible para tener una experiencia positiva y controlada del vuelo del drone.\\

Esta conexión deberá realizarse entre el ordenador local y un segundo ordenador, que será desde el que el usuario podrá teleoperar el vehículo.\\

Durante toda la memoria nos referiremos como ordenador local, par local o navegador local al dispositivo que usaremos para conectarnos al cuadricoptero y el cuál deberá ir a bordo del drone, y ordenador remoto, par remoto o navegador remoto al ordenador desde el cuál teleoperaremos el vehículo.\\

Como sub-objetivos para esta conexión tenemos los siguientes:

\begin{itemize}

\item \emph{Visualización de cámara a bordo:} La conexión tiene que ser capaz de transportar el flujo audiovisual de la cámara del drone desde el navegador local hasta el navegador remoto.

\item \emph{Sensores de navegación.} Los datos obtenidos de los sensores de navegación del cuadricoptero deberán ser enviados al ordenador remoto donde se visualizaran.

\item \emph{Localización espacial del drone.}La brújula y GPS del cuadricóptero deberán ser recogidas por el ordenador local, y deberán enviarse al ordenador remoto para poder visualizar en este un mapa para tener localizada la posición del vehículo.

\item \emph{Ordenes}. Desde el ordenador remoto deberán enviarse hacia el ordenador local las ordenes dadas para el aterrizaje, despegue, y comandos de movimiento.


\end{itemize}

De manera general el par local deberá establecer la conexión con el drone, acceder a sus sensores y a su cámara. Deberá también establecer una conexión con el ordenador remoto y enviarle todos estos datos obtenidos del drone. Del par remoto recibirá las ordenes y comandos de movimiento, que deberá enviárselos al drone.\\

El par remoto deberá establecer la conexión con el par local. De este recibirá todos los datos del drone y deberá representarlos de una manera que el usuario final pueda conocer la situación de vuelo del drone en cada momento. Deberá tener una interfaz amigable que le permita recoger las órdenes de movimiento dadas por el usuario, y enviárselas al par local.\\

En la figura \ref{fig:esquemageneral} tenemos un esquema de la estructura general del proyecto.\\

\begin{figure}[htb]
\centering
\includegraphics[width=0.9\textwidth]{conexiones}
\caption{Esquema general del proyecto}
\label{fig:esquemageneral}
\end{figure}

\subsection{Interfaz amigable y actual}

Uno de los requisitos es desarrollar una interfaz clara, que nos muestre de una manera lo mas realista, clara y concisa posible los datos recogidos de los sensores del drone y la cámara para permitirnos conocer el estado de vuelo del drone en cada momento: altitud, inclinación, velocidad...\\


Para que la satisfacción de vuelo sea satisfactoria la interfaz tiene que tener unos controles de movimiento que sean sencillos, simples e intuitivos.\\

Esta interfaz deberá realizarse con tecnología web actual y moderna, y deberá permitirnos el manejo del cuadricóptero de la forma mas realista y similar posible a los sistemas de control de drones que hemos hablado en el capítulo \ref{cap:controldrones}.\\

\section{Metodología y plan de trabajo}

La realización de un proyecto requiere de una metodología que establezca las pautas a seguir y la planificación de las tareas que se deben llevar a cabo para cumplir los objetivos del mismo. Hemos escogido el modelo de \emph{desarrollo en espiral}, ya que es un modelo ampliamente usado en la ingeniería de \emph{software}. Este modelo define una serie de ciclos que se repiten en un bucle hasta el final del proyecto, dividiéndolo en varias subtareas más sencillas y estableciendo puntos de control al final de cada iteración en los que se evalúa el trabajo realizado y se enfocan las nuevas tareas para continuar.\\

Esta metodología recibe su nombre por la forma de espiral que tiene su representación gráfica o diagrama de flujo, que podemos ver en la figura \ref{fig:planificacion_espiral}. En cada iteración se llevan a cabo las siguientes actividades:

\begin{itemize}
 \item \textbf{Determinar los objetivos}, dividir en subobjetivos y fijar requisitos.
 \item \textbf{Analizar los riesgos} y factores que impidan o dificulten el trabajo y las consecuencias negativas que este
 pueda ocasionar.
 \item \textbf{Desarrollar} las tareas para lograr los objetivos según los requisitos especificados.
 \item \textbf{Planificar} las próximas fases tras evaluar el transcurso del proyecto.
\end{itemize}

\begin{figure}[htb]
\centering
\includegraphics[width=0.9\textwidth]{espiral}
\caption{Esquema general del proyecto}
\label{fig:planificacion_espiral}
\end{figure}

Durante el ciclo de vida del proyecto se han hecho reuniones periódicas con el tutor. En ellas se evaluaban los avances logrados y se marcaba la hoja de ruta a tomar para los siguientes días de desarrollo. Si los puntos marcados en sesiones anteriores no se habian finalizado se ampliaba el plazo o se intentaba buscar otra manera de avance.\\

Para facilitar el seguimiento del proyecto se ha hecho uso de un mediawiki\footnote{http://jderobot.org/Irodmar-tfg} de JdeRobot en el que se iba actualizando cada avance que se lograba, con explicaciones y vídeos e imágenes. Se ha utilizado la plataforma Github\footnote{https://github.com/RoboticsURJC-students/2015-tfg-irodmar}, que es un sistema de control de versiones para alojar el código de todo el proyecto.\\

El plan de trabajo para todo el proyecto se puede dividir en las siguientes etapas:

\begin{itemize}
\item \textbf{Familiarización con JdeRobot}: Primer contacto con esta plataforma y sus herramientas para conocer su funcionamiento.
\item \textbf{Aprendizaje de tecnologías web necesarias:} Conocer las tecnologías web que van a ser necesarias para el desarrollo del proyecto. Entre ellas se encuentra WebRTC, HTML5, CSS3, WebGL, ThreeJS, o jQuery. Primer contacto también con el \emph{middleware} ICE.
\item \textbf{Desarrollo de la conexión local:} Creacion de toda la infraestructura necesaria para la interconexión entre el navegador local y el drone.
\item \textbf{Desarrollo conexión remota}: Desarrollo de la conexión remota que interconectara los dos pares.
\item \textbf{Desarrollo interfaz}: Desarrollo de la interfaz amigable para teleoperar el drone.
\item \textbf{Experimentos}: Primero se realizan pruebas con el simulador, y cuando el código este suficientemente maduro se prueba con un drone real.
\end{itemize}






